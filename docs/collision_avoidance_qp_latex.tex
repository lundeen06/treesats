% Minimum-fuel collision avoidance: convex QP formulation
% Corresponds to sat/control/collision_avoidance.py

\documentclass[11pt]{article}
\usepackage{amsmath,amssymb,bm}
\usepackage[margin=1in]{geometry}

\title{Minimum-Fuel Collision Avoidance: Convex QP}
\author{}
\date{}

\begin{document}
\maketitle

\section{Setup}
\begin{itemize}
  \item Chief satellite: initial state $(\bm{r}_c(0), \bm{v}_c(0))$ in ECI (position and velocity).
  \item Constellation: deputies $k = 1,\ldots,K$ with known ECI positions over time.
  \item Nominal trajectories: propagate chief and constellation over one orbit to get $\bm{r}_c(t)$, $\bm{p}_k(t)$ in ECI [km].
  \item Impulsive maneuver at $t=0$: apply $\Delta\bm{v} \in \mathbb{R}^3$ (ECI, km/s). New velocity is $\bm{v}_c(0) + \Delta\bm{v}$; position at $t=0$ unchanged.
\end{itemize}

\section{Linearized effect of $\Delta\bm{v}$}
The sensitivity of chief position at time $t$ to initial velocity is the $3\times3$ matrix
\[
  \bm{\Phi}_{rv}(t) = \frac{\partial \bm{r}_c(t)}{\partial \bm{v}_c(0)}.
\]
To first order, the chief position after the maneuver is
\[
  \bm{r}_c^{\mathrm{new}}(t) = \bm{r}_c(t) + \bm{\Phi}_{rv}(t) \, \Delta\bm{v}.
\]
(Deputy positions are unchanged.)

\section{Distance constraint (linearized)}
For deputy $k$ at time $t$, the relative position is
\[
  \bm{r}_{\mathrm{rel}}(t,k) = \bm{r}_c(t) - \bm{p}_k(t), \qquad d(t,k) = \|\bm{r}_{\mathrm{rel}}(t,k)\|, \qquad \bm{u}(t,k) = \frac{\bm{r}_{\mathrm{rel}}(t,k)}{d(t,k)}.
\]
After the maneuver, the new relative position is
\[
  \bm{r}_{\mathrm{rel}}^{\mathrm{new}} = \bm{r}_{\mathrm{rel}} + \bm{\Phi}_{rv}(t) \, \Delta\bm{v}.
\]
The first-order change in distance along the line of sight $\bm{u}$ is
\[
  \Delta d \approx \bm{u}^\top \bm{\Phi}_{rv}(t) \, \Delta\bm{v}.
\]
We require the new distance to be at least $r_{\min}$ (the minimum allowed separation, in km):
\[
  d + \Delta d \geq r_{\min} \quad \Leftrightarrow \quad \bm{u}^\top \bm{\Phi}_{rv}(t) \, \Delta\bm{v} \geq r_{\min} - d.
\]
Define one linear inequality per constrained $(t,k)$:
\[
  \bm{a}_{i}^\top \Delta\bm{v} \geq b_i, \qquad \bm{a}_i = \bm{\Phi}_{rv}(t)^\top \bm{u}(t,k), \qquad b_i = r_{\min} - d(t,k).
\]

\section{Convex QP (minimum fuel)}
We minimize the squared magnitude of the impulse (minimum fuel in the 2-norm sense) subject to the linearized safety constraints:
\begin{equation}
\boxed{
  \begin{aligned}
    \min_{\Delta\bm{v} \in \mathbb{R}^3} \quad & \|\Delta\bm{v}\|_2^2 \\
    \text{s.t.} \quad & \bm{A} \Delta\bm{v} \geq \bm{b}.
  \end{aligned}
}
\end{equation}
Here, $\bm{A}$ has one row $\bm{a}_i^\top$ per constrained pair $(t,k)$, and $\bm{b}$ is the vector of $b_i$. The problem is a convex quadratic program (quadratic objective, linear inequality constraints) and is solved with a QP solver (e.g., CVXPY).

\section{Constraint selection}
Constraints are not added at every $(t,k)$. For each deputy $k$, they are added at the time of \emph{nominal} closest approach $t_{\min}(k)$ and neighboring time steps, and only when the nominal distance $d(t,k) < \text{\texttt{constraint\_margin\_km}}$. After solving, the trajectory $(\bm{r}_c(0), \bm{v}_c(0)+\Delta\bm{v})$ is propagated and checked at all $(t,k)$; if any distance is below $r_{\min}$, additional constraints at those $(t,k)$ are added and the QP is re-solved (verification loop).
\end{document}
